\begin{SCn}

\scnsectionheader{\currentname}

\scnstartsubstruct

\scnheader[\scnidtf{Предпосылки создания компьютерных систем нового поколения}]{Предметная область кибернетических систем}
\scniselement{предметная область}
\scnsdmainclasssingle{кибернетическая система}
\scnsdclass{искусственная сущность; компьютерная система;простая кибернетическая система;индивидуальная кибернетическая система;кибернетическая система, встроенная в индивидуальную кибернетическую систему;многоагентная система;одноуровневая многоагентная система;коллектив индивидуальных кибернетических систем;иерархический коллектив индивидуальных кибернетических систем;информация, хранимая в памяти кибернетической системы;абстрактная память кибернетической системы;решатель задач кибернетической системы;действие кибернетической системы;задача;задача, решаемая кибернетической системой;навык; интерфейс кибернетической системы;физическая оболочка кибернетической системы;память кибернетической системы;процессор кибернетической системы;компьютер;качество кибернетической системы;гибридная кибернетическая система;приспособленность кибернетической системы к её совершенствованию;гибкость кибернетической системы;производительность кибернетической системы;надежность кибернетической системы;качество физической оболочки кибернетической системы;качество памяти кибернетической системы;интеллект;образованность кибернетической системы;интеллектуальная система; кибернетическая система, основанная на знаниях;кибернетическая система, управляемая знаниями;целенаправленная кибернетическая система;обучаемая кибернетическая система; социально ориентированная кибернетическая система;интеллектуальная компьютерная система;информация;сенсорная информация;качество решателя задач кибернетической системы;обучаемость кибернетической системы;стратифицированность кибернетической системы;рефлексивность кибернетической системы;синергетическая кибернетическая система;социализация кибернетической системы}
\scnsdrelation{информация, хранимая в памяти кибернетической системы*; задача, решаемая кибернетической системой*; внешняя среда кибернетической системы*;среда кибернетической системы*;агент*}

\scnidtf{Иерархическая система свойств (характеристик) кибернетических систем, определяющих общий (интегральный) уровень их качества}
\scnidtf{Эволюционный подход к определению качества и, в частности, уровня интеллекта кибернетической системы}

\scntext{аннотация}{Рассмотрена иерархическая система свойств (в т.ч. способностей) кибернетических систем, определяющих их качество и позволяющих сформулировать требования, которым должна удовлетворять высокоинтеллектуальная система (кибернетическая система с сильным интеллектом).

Уровень качества кибернетических систем определяется достаточно большим набором свойств (параметров, характеристик) кибернетических систем, каждое из которых определяет уровень качества кибернетической системы в соответствующем аспекте (ракурсе), указывая (задавая) уровень развития конкретных  способностей и возможностей кибернетической системы. При этом важно подчеркнуть следующее:
\begin{scnitemize}
	\item существенное значение имеет не столько сам набор свойств, а иерархия этих свойств, позволяющая уточнять (детализировать) направления проявления (реализации) каждого свойства;
	\item существенное значение также имеет \uline{баланс} уровней развития различных свойств -- вклад разных свойств, обеспечивающих (определяющих) значение одного и того же свойства более высокого уровня иерархии, а значение этого свойства более высокого уровня может быть разным. Из этого следует, что не всегда следует акцентировать внимание на развитие некоторых свойств (характеристик). Нужен целостный, коллективный подход;
	\item рассмотренная иерархия свойств кибернетических систем является общей как для естественных, так и для искусственных кибернетических систем;
	\item приведенная иерархическая детализация свойств кибернетических систем (с помощью отношения ``\textit{частное свойство*}'' и отношения ``\textit{свойство-предпосылка*}''), определяющих качество таких систем, (1) дает возможность четко определить направления совершенствования (развития) кибернетических систем и (2) дает ориентир (систему критериев) для обоснования конкретных предложений по совершенствованию компьютерных систем, а также для сравнения различных альтернативных предположений;
	\item особое значение для развития кибернетических систем имеют такие их свойства, как стратифицированность, рефлексивность и социализация;
	\item важное значение имеет не только совершенствование кибернетических систем в соответствии с иерархической системой их свойств, но и совершенствование (в том числе, детализация) самой этой иерархической системы свойств. 
\end{scnitemize}}

\scntext{предисловие}{Свойства (способности), которым должны удовлетворять \textit{интеллектуальные системы}, рассматриваются в целом ряде публикаций. Тем не менее, для \uline{практической} реализации \textit{компьютерных систем}, обладающих указанными свойствами (способностями), т.е. \textit{интеллектуальных компьютерных систем}, необходимо детализировать (уточнить) эти \textit{свойства}, пытаясь свести их к более конструктивным, прозрачным и понятным для реализации свойствам.}

\scnrelfromset{рассматриваемые вопросы}{
\scnfileitem{По каким свойствам (параметрам, характеристикам, способностям) кибернетических систем можно оценивать уровень их качества.};
\scnfileitem{Можно ли считать уровень развития какого-либо свойства (способности) кибернетической системы, т.е. значение какого-либо ее параметра (характеристики) оценкой уровня качества кибернетической системы по соответствующему аспекту.};
\scnfileitem{Может ли какое-либо свойство кибернетических систем определять (влиять на) значение сразу нескольких свойств более высокого уровня иерархии.};
\scnfileitem{Какими отношениями свойства кибернетических систем связаны со свойствами более низкого и, соответственно, более высокого уровня иерархии.};
\scnfileitem{Зачем нужна такая иерархия свойств, определяющих качество кибернетических систем и позволяющих детализировать (уточнять) то, какими свойствами определяется уровень (степень) развития каждого свойства (значение каждого свойства) за исключением свойств, которые условно можно считать элементарными, не требующими детализации (по крайнем мере, пока).};
\scnfileitem{Может ли иерархия свойств, определяющих качество кибернетических систем, быть критерием оценки и выбора того или иного подхода к построению интеллектуальных компьютерным систем.};
\scnfileitem{Какими свойствами (способностями) должна обладать кибернетическая система, имеющая высокий уровень интеллекта.};
\scnfileitem{Какими свойствами определяется уровень интеллекта многоагентной кибернетической системы.};
\scnfileitem{Как связан уровень интеллекта многоагентной системы с уровнем интеллекта агентов, входящих в ее состав.};
\scnfileitem{Почему, например, не каждый коллектив высокоинтеллектуальных людей демонстрирует высокий уровень интеллекта самого коллектива.};
\scnfileitem{Какими дополнительными свойствами кроме достаточно высокого уровня интеллекта должны обладать агенты многоагентных систем для обеспечения высокого уровня интеллекта самой многоагентной системы как самостоятельной целостной кибернетической системы.};
\scnfileitem{Как зависит уровень интеллекта многоагентной системы от организации взаимодействия между агентами, например, от использования централизованного или децентрализованного управления.}}

\scnrelfromvector{ключевые знаки}{
	кибернетическая система\\
	\scnaddlevel{1}	
	\scnsubdividing{
		естественная кибернетическая система;
		компьютерная система
		\scnaddlevel{1}	
		\scnidtf{искусственная кибернетическая система}
		\scnaddlevel{-1};
		естественно-искусственная кибернетическая система
		\scnaddlevel{1}
		\scnidtf{кибернетическая система, являющаяся симбиозом компонентов как естественного, так и искусственного происхождения}
		\scnaddlevel{-1}}
	\scnaddlevel{-1};
	качество кибернетической системы;
	физическая оболочка кибернетической системы;
	качество физической оболочки кибернетической системы;
	интеллект
	\scnaddlevel{1}
	\scnidtf{уровень интеллекта кибернетической системы}
	\scnidtf{интеллектуальность}
	\scnaddlevel{-1};
	интеллектуальная система
	\scnaddlevel{1}
	\scnidtf{интеллектуальная кибернетическая система}
	\scnsuperset{интеллектуальная компьютерная система}
	\scnaddlevel{-1};
	информация, хранимая в памяти кибернетической системы;
	качество информации, хранимой в памяти кибернетической системы;
	база знаний;
	смысловое представление информации в памяти кибернетической системы;
	решатель задач кибернетической системы;
	качество решателя задач кибернетической системы;
	память кибернетической системы;
	качество памяти кибернетической системы;
	обучаемость кибернетической системы;
	гибкость кибернетической системы;
	стратифицированность кибернетической системы;
	рефлексивность кибернетической системы
	\scnaddlevel{1}
	\scnidtf{уровень рефлексии кибернетической системы}
	\scnaddlevel{-1};
	многоагентная система;
	качество многоагентной системы;
	унифицированность агентов многоагентной системы;
	семантическая совместимость агентов многоагентной системы;
	социализация кибернетической системы
	\scnaddlevel{1}
	\scnidtf{способность кибернетической системы своей внутренней и внешней деятельностью обеспечивать высокий уровень интеллекта тех многоагентных систем, членом (агентом) которых она является}
	\scnaddlevel{-1}}

\scnrelfromvector{библиография}{
	\scncite{Viner1952};
	\scncite{Pospelov1989};
	\scncite{Finn2008};
	\scncite{YarushinaHS};
	\scncite{RedkoV2019}}

\newpage
\scnreltovector{конкатенация сегментов}{
	Уточнение понятия кибернетической системы;
	Комплекс свойств, определяющий общий уровень качества кибернетической системы;
	Комплекс свойств, определяющих качество физической оболочки кибернетической системы;	
	Комплекс свойств, определяющих уровень интеллекта кибернетической системы;	
	Комплекс свойств, определяющий качество информации, хранимой в памяти кибернетической системы
	;Комплекс свойств, определяющих качество решателя задач кибернетической системы
	;Комплекс свойств, определяющих уровень обучаемости кибернетической системы
	;Комплекс свойств, определяющих качество многоагентной системы
	;Комплекс свойств, определяющих уровень социализации кибернетической системы как фактора существенного повышения уровня ее обучаемости, а также фактора существенного повышения качества всех тех многоагентных систем, в состав которых входит данная кибернетическая система}

\input{Contents/chapter1/intro_hs/intro_hs_segment1}

\input{Contents/chapter1/intro_hs/intro_hs_segment2}

\input{Contents/chapter1/intro_hs/intro_hs_segment3}

\input{Contents/chapter1/intro_hs/intro_hs_segment4}

\input{Contents/chapter1/intro_hs/intro_hs_segment5}

\input{Contents/chapter1/intro_hs/intro_hs_segment6}

\input{Contents/chapter1/intro_hs/intro_hs_segment7}

\input{Contents/chapter1/intro_hs/intro_hs_segment8}

\input{Contents/chapter1/intro_hs/intro_hs_segment9}

\bigskip
\scnendstruct \scnendcurrentsectioncomment
\end{SCn}
